% Use Frédéric Santos' template as a starting point for the beamer presentation.
\documentclass[presentation]{beamer}
\beamertemplatenavigationsymbolsempty{}
\usepackage[utf8]{inputenc}
\usepackage[T1]{fontenc}
\usepackage{graphicx}
\graphicspath{ {../resources/} }
\usepackage{hyperref}
\usepackage{color}

\usepackage{newverbs}



\usetheme{CambridgeUS}
\usepackage[english]{babel}
\usepackage{float}
\definecolor{PalePurple}{RGB}{127, 90, 182}
\definecolor{DarkPurple}{RGB}{98, 36, 134}
\definecolor{grey}{RGB}{51, 63, 72}
\setbeamercolor{title}{fg=white, bg=DarkPurple}
\setbeamercolor{frametitle}{fg=black}
\setbeamercolor{structure}{fg=PalePurple}
\setbeamercolor{section in head/foot}{fg=white, bg=PalePurple}
\setbeamercolor{subsection in head/foot}{fg=DarkPurple}
\setbeamercolor{title in head/foot}{fg=white, bg=DarkPurple}
\setbeamercolor{date in head/foot}{fg=grey}
\setbeamercolor{block title}{fg=white, bg=DarkPurple}
\setbeamercolor{block body}{bg=gray!20}


\hypersetup{
  colorlinks=true,
  urlcolor=DarkPurple
}


\newverbcommand{\bverb}
{\begin{lrbox}{\verbbox}}
  {\end{lrbox}\colorbox{PalePurple!40}{\box\verbbox}}

%% Structure of a slide :
\setbeamertemplate{footline}
{
\leavevmode%
\hbox{%
\begin{beamercolorbox}[wd=.75\paperwidth,ht=2.25ex,dp=1ex,center]{title in head/foot}%
\usebeamerfont{author in head/foot}
\end{beamercolorbox}%
\begin{beamercolorbox}[wd=.25\paperwidth,ht=2.25ex,dp=1ex,center]{date in head/foot}%
\hspace*{1ex}
\end{beamercolorbox}}%
}
\author{Alexander E. Zarebski}
\date{\today}
\title{ESS with Spacemacs}
\begin{document}

\maketitle

\begin{frame}
  \frametitle{Questions}
  \begin{itemize}
  \item What is Spacemacs?
  \item How do I get started?
  \item Where can I learn more?
  \end{itemize}
\end{frame}

\begin{frame}
  \frametitle{What is Spacemacs?}
  
  \begin{figure}
    \centering
    \includegraphics[width=.7\linewidth]{spacemacs-ess-demo}
  \end{figure}
  
  Spacemacs is a configuration starter kit, designed upon four core pillars.
\end{frame}

\begin{frame}
  \frametitle{Core pillars}
 \begin{itemize}
 \item Mnemonic
 \item Discoverable
 \item Consistent
 \item ``Crowd-Configured''
 \end{itemize} 
\end{frame}


\begin{frame}[fragile]
  \frametitle{Mnemonic and discoverable}

  \begin{itemize}
  \item \bverb|SPC f| file management eg \bverb|SPC f f| for \emph{find file}
  \item \bverb|SPC m| major mode commands eg \bverb|SPC m s b| to \emph{send buffer} to REPL
  \item \bverb|SPC w/b/g| for window/buffer/magit management\dots
  \end{itemize} 
\end{frame}


  \begin{frame}
    \frametitle{Consistent}
    Functionality is largely language agnostic, so you only need to learn the
    keys once.
  \end{frame}


  \begin{frame}
    \frametitle{Crowd-configured}
    You don't have time to get the perfect emacs configuration, so let the
    community help you!
  \end{frame}


  \begin{frame}
    \frametitle{Layers}
    Functionality is implemented in \emph{layers}, eg there is an ESS layer and
    a Python layer and a magit layer, and you can build new ones by stacking
    existing layers.
  \end{frame}

  
  \begin{frame}
    \frametitle{Quick start}
    There is a quick start guide in the documentation and an ESS specific
    \href{https://github.com/ess-intro/presentation-ess-from-spacemacs/tree/main/presentation}{guide}
    is included in this repository.
  \end{frame}

  \begin{frame}
    \frametitle{Next steps}
    More information can be found all over the web
    \begin{itemize}
    \item \href{https://www.spacemacs.org/}{Spacemacs website}
    \item \href{https://github.com/syl20bnr/spacemacs}{Github}
    \item \href{https://gitter.im/syl20bnr/spacemacs}{Gitter}
    \item there is even a \href{https://www.reddit.com/r/spacemacs/}{subreddit}
    \end{itemize}
  \end{frame}
  
\end{document}


